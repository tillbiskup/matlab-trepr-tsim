%%%%%%%%%%%%%%%%%%%%%%%%%%%%%%%%%%%%%%%%%%%%%%%%%%%%%%%%%%%%%%%%%%%%%%%%%%%%%%%%%%%%%%%%%%
%
% TsimFitReport-en.tex - LaTeX template for reporting fit results from the Tsim toolbox.
%
%%%%%%%%%%%%%%%%%%%%%%%%%%%%%%%%%%%%%%%%%%%%%%%%%%%%%%%%%%%%%%%%%%%%%%%%%%%%%%%%%%%%%%%%%%
%
% The "-en" indicates that this is a ENGLISH template. Feel free to translate into any
% other language while containing the general structure and contents.
%
% This template is meant to be used with the report functionality of the Tsim toolbox
% in combination with the common toolbox and the tpl template engine.
%
% Make sure to set the delimiters for fields to "[[" and "]]" accordingly, as it turned 
% out that the default "{{" and "}}" delimiters are not a very good idea for use with
% LaTeX.
%
% Please note that this template makes use of knowledge about some internals of the tpl
% template engine. It may serve as starting point for own templates for use with tpl, but
% be aware that tpl may still change quite dramatically before it reaches a stable
% interface.
%
% For LaTeX purists: This template doesn't use much of the structures LaTeX provides,
% such as sections and title. In the future there might be a LaTeX class for those reports
% helping with a common look and feel.
%
% All packages used should be fairly common and included in a standard LaTeX installation.
%
% A NOTE TO TEMPLATE DEVELOPERS:
%
% If using this template as starting point for a series of similar templates, it might be
% wise to use the "include" hook of the tpl template engine and to separate large parts of
% the header in an additional document that just gets added to all templates.
%
%%%%%%%%%%%%%%%%%%%%%%%%%%%%%%%%%%%%%%%%%%%%%%%%%%%%%%%%%%%%%%%%%%%%%%%%%%%%%%%%%%%%%%%%%%

% Copyright (c) 2015-20, Till Biskup, Deborah Meyer
% 2020-03-23

%%%%%%%%%%%%%%%%%%%%%%%%%%%%%%%%%%%%%%%%%%%%%%%%%%%%%%%%%%%%%%%%%%%%%%%%%%%%%%%%%%%%%%%%%%
% HEADER
%%%%%%%%%%%%%%%%%%%%%%%%%%%%%%%%%%%%%%%%%%%%%%%%%%%%%%%%%%%%%%%%%%%%%%%%%%%%%%%%%%%%%%%%%%
\documentclass{article}
\usepackage[paper=a4paper,left=25mm,right=25mm,top=25mm,bottom=20mm]{geometry}
\usepackage[utf8]{inputenc}
\usepackage[T1]{fontenc}

% For English reports, load appropriate language package
\usepackage[british]{babel}

% Links within the PDF and alike
\usepackage{hyperref}

% Nice looking tables
\usepackage{booktabs}
% As "caption" shall be typeset above a table, no separation between caption and table.
\setlength{\abovetopsep}{1em}

% Math packages
\usepackage{amsmath}

% Graphics handling
\usepackage{graphicx}
  \DeclareGraphicsExtensions{.pdf,.jpg,.png}

% Headers and footers
\usepackage{fancyhdr}
\fancyhead[L]{Triplet Simulation of [[@sample.name]]}
\pagestyle{fancy}

% Change indentation and separation between paragraphs
\setlength{\parindent}{0ex}
\setlength{\parskip}{1ex}

% Define some basic commands
\usepackage{textcomp}
\newcommand{\matlab}{\textsf{MATLAB\textsuperscript{\textregistered}}}

\newcommand*{\command}[1]{\texttt{#1}}

% Caption setup
\makeatletter

\setlength\abovecaptionskip{5\p@}
\setlength\belowcaptionskip{0\p@}
\@ifundefined{captionfontsize}{%
  \def\captionfontsize{\small} }{}
\long\def\@makecaption#1#2{%
  \vskip\abovecaptionskip\captionfontsize
  \sbox\@tempboxa{\textbf{#1:} #2}%
  \ifdim \wd\@tempboxa >\hsize
    \textbf{#1:} #2\par
  \else
    \global \@minipagefalse
    \hb@xt@\hsize{\hfil\box\@tempboxa\hfil}%
  \fi
  \vskip\belowcaptionskip}
  
\makeatother


%%%%%%%%%%%%%%%%%%%%%%%%%%%%%%%%%%%%%%%%%%%%%%%%%%%%%%%%%%%%%%%%%%%%%%%%%%%%%%%%%%%%%%%%%%
% BODY
%%%%%%%%%%%%%%%%%%%%%%%%%%%%%%%%%%%%%%%%%%%%%%%%%%%%%%%%%%%%%%%%%%%%%%%%%%%%%%%%%%%%%%%%%%
\begin{document}
\thispagestyle{empty}

\vspace*{-1.5cm}

\noindent\rule[1.5ex]{\textwidth}{1pt}

\begin{sffamily}\bfseries\large
Results of the triplet simulation of [[@sample.name]]
\end{sffamily}

\noindent\rule{\textwidth}{1pt}

\begin{flushright}\slshape
--- [[@history{1}.system.username]], [[@history{1}.date]]
\end{flushright}

\vspace*{1.5em}

A triplet simulation has been fitted to the TREPR spectrum of [[@sample.name]] using the program \command{TSim} and the simulation routine \command{[[@Tsim.sim.routine]]}.

[[if ~isempty(this.assignments.Tsim.remarks.purpose)]]
\textbf{Purpose:} [[@Tsim.remarks.purpose]]
[[end]]

[[if ~isempty(this.assignments.Tsim.remarks.comment)]]
\textbf{User comment:} [[@Tsim.remarks.comment]]
[[end]]

For a first overview of the results of the triplet simulation and the fit to the experimental data, see Fig.~\ref{fig:results}. For the simulation parameters, see Tab.~\ref{tab:simparameter}, p.~\pageref{tab:simparameter}, and for the fitting parameters, see, Tab.~\ref{tab:fitparameter}, p.~\pageref{tab:fitparameter}.


\begin{figure}[h]
\centering
[[if ~isempty(this.assignments.Tsim.results.figureFileName)]]
\includegraphics[width=\textwidth]{[[@Tsim.results.figureFileName]]}
[[end]]
\caption{\textbf{Recorded data together with a fitted simulation.} The simulation has been fitted to a slice at  [[@Tsim.fit.spectrum.sectionCenter|f=5.2e]]~s[[if ~isempty(this.assignments.Tsim.fit.spectrum.sectionWidth)]] (averaged over [[@Tsim.fit.spectrum.sectionWidth|f=5.2e]]~s)[[end]]. Experimental parameters: microwave frequency [[@parameters.bridge.MWfrequency.value|f=8.5f]]~[[@parameters.bridge.MWfrequency.unit]], microwave power [[@parameters.bridge.power.value|f=5.2f]]~[[@parameters.bridge.power.unit]] ([[@parameters.bridge.attenuation.value|f=2.0f]]~[[@parameters.bridge.attenuation.unit]]), [[@parameters.recorder.averages|f=4.0f]] spp, light excitation at  [[@parameters.laser.wavelength.value|f=3.0f]]~[[@parameters.laser.wavelength.unit]] with [[@parameters.laser.power.value|f=3.0f]]~[[@parameters.laser.power.unit]] pulse power. For simulation parameters, see Tab.~\ref{tab:simparameter}, p.~\pageref{tab:simparameter}.}
\label{fig:results}
\end{figure}

\begin{center}
\setlength{\fboxsep}{1.5ex}\setlength{\fboxrule}{.75pt}
\fbox{%
\begin{minipage}{.9\textwidth}\centering
\textbf{--- Disclaimer for the simulation routine ---}
\vspace*{1ex}

\slshape
[[foreach @Tsim.acknowledgement.sim]]
[[@Tsim.acknowledgement.sim]]\par
[[end]]
\end{minipage}
}
\end{center}

\clearpage


Fitting has been done using the function \command{lsqcurvefit} within \matlab{}. For the fit options used see Tab.~\ref{tab:fitopt}. The exit message of the fit was as follows:

\begin{quote}\small
\ttfamily\raggedright
[[@Tsim.fit.fitreport.exitmessage]]
\end{quote}

Recorded data and simulation have each been normalised to the same area for the fitting.[[if ~isempty(this.assignments.Tsim.fit.weighting.weightingArea)]] Additionally, the following weightings for certain areas of the experimental spectrum have been used:

\begin{table}[h]
\caption{\textbf{Overview of the weighting areas and their respective weights.} The weightings used during fitting to force the optimisation algorithm into a respective minimum.}
\label{tab:weights}
\centering
\begin{tabular}{cc}
\toprule
\textbf{Area} & \textbf{Weight}
\\
\midrule 

[[foreach element of @Tsim.fit.weighting.weightingFactor]]
[[@Tsim.fit.weighting.weightingArea(2*[[@element]]-1)]]--[[@Tsim.fit.weighting.weightingArea(2*[[@element]])]] mT      & [[@Tsim.fit.weighting.weightingFactor([[@element]])]]
\\

[[end]]

\bottomrule
\end{tabular}
\end{table}
[[end]]

\begin{table}[h]
\caption{\textbf{Overview of the simulation parameters.} The simulation parameters are the complete set of parameters used for simulating the spectrum shown in Fig.~\ref{fig:results}. $\Gamma$ is the line width the simulation has been convoluted with. See Tab.~\ref{tab:fitparameter} for the parameters that have been varied for the fitting.}
\label{tab:simparameter}
\centering
\begin{tabular}{cccccccc} 
\toprule
\multicolumn{8}{c}{\textbf{Standard parameters} }
\\
\midrule 
$g_x$      & $g_y$      & $g_z$      & $p_1$   & $p_2$   & $p_3$   & $D$ / MHz & $E$ / MHz \\

[[@Tsim.sim.simpar.gx|f=7.5f]] & [[@Tsim.sim.simpar.gy|f=7.5f]] & [[@Tsim.sim.simpar.gz|f=7.5f]] & [[@Tsim.sim.simpar.p1|f=5.3f]] & [[@Tsim.sim.simpar.p2|f=5.3f]] & [[@Tsim.sim.simpar.p3|f=5.3f]] & [[@Tsim.sim.simpar.D|f=7.1f]]  & [[@Tsim.sim.simpar.E|f=7.1f]]     \\

\bottomrule
\end{tabular}

\begin{tabular}{cc}
\toprule
\textbf{Parameter} & \textbf{Wert}
\\
\midrule

[[if isfield(this.assignments.Tsim.sim.simpar,'lwGauss')]]
$\Gamma_\text{Gauss}$ / mT & $[[@Tsim.sim.simpar.lwGauss]]$
\\
[[end]]
[[if isfield(this.assignments.Tsim.sim.simpar,'lwLorentz')]]
$\Gamma_\text{Lorentz}$ / mT & $[[@Tsim.sim.simpar.lwLorentz]]$
\\
[[end]]
[[if isfield(this.assignments.Tsim.sim.simpar,'DeltaB')]]
$\Delta B$ / mT & $[[@Tsim.sim.simpar.DeltaB]]$
\\

[[end]]
\bottomrule
\end{tabular}
\end{table}

\begin{table}[h]
\caption{\textbf{Overview of the fitting parameters.} These parameters are usually a subset of the parameters used for simulating the spectrum shown in Fig.~\ref{fig:results}.  For a complete set of simulation parameters see Tab.~\ref{tab:simparameter}. The error given for each parameter is the standard deviation obtained from the Jacobi matrix.}
\label{tab:fitparameter}
\centering
\begin{tabular}{cccccc}
\toprule
\textbf{Parameter} & \textbf{Start value} & \textbf{Lower boundary} & \textbf{Upper boundary} & \textbf{Endwert} & \textbf{Error}
\\
\midrule

[[foreach element of @Tsim.fit.fitpar]]
\command{[[@Tsim.fit.fitpar([[@element]])]]} & [[@Tsim.fit.initialvalue([[@element]])]] & [[@Tsim.fit.lb([[@element]])]] & [[@Tsim.fit.ub([[@element]])]] & [[@Tsim.fit.finalvalue([[@element]])]] & [[@Tsim.fit.fitreport.stdDev([[@element]])]]
\\

[[end]]

\bottomrule
\end{tabular}
\end{table}

\begin{table}[h]
\caption{\textbf{Overview of the options the \matlab{} routine \command{lsqcurvefit} has been called with.} See also the exit criteria for the routine \command{lsqcurvefit} given above.}
\label{tab:fitopt}
\centering
\begin{tabular}{lr}
\toprule
\textbf{Parameter} & \textbf{Wert}
\\
\midrule

[[foreach @Tsim.fit.fitopt]]
\command{[[@Tsim.fit.fitopt.fieldname]]} & [[@Tsim.fit.fitopt.fieldvalue|f=4.2e]]
\\

[[end]]
\bottomrule
\end{tabular}
\end{table}


\end{document}