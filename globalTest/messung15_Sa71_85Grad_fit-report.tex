%%%%%%%%%%%%%%%%%%%%%%%%%%%%%%%%%%%%%%%%%%%%%%%%%%%%%%%%%%%%%%%%%%%%%%%%%%%%%%%%%%%%%%%%%%
%
% TsimFitReport-de.tex - LaTeX template for reporting fit results from the Tsim toolbox.
%
%%%%%%%%%%%%%%%%%%%%%%%%%%%%%%%%%%%%%%%%%%%%%%%%%%%%%%%%%%%%%%%%%%%%%%%%%%%%%%%%%%%%%%%%%%
%
% The "-de" indicates that this is a GERMAN template. Feel free to translate into any
% other language while containing the general structure and contents.
%
% This template is meant to be used with the report functionality of the Tsim toolbox
% in combination with the common toolbox and the tpl template engine.
%
% Make sure to set the delimiters for fields to "" and "" accordingly, as it turned 
% out that the default "{{" and "}}" delimiters are not a very good idea for use with
% LaTeX.
%
% Please note that this template makes use of knowledge about some internals of the tpl
% template engine. It may serve as starting point for own templates for use with tpl, but
% be aware that tpl may still change quite dramatically before it reaches a stable
% interface.
%
% For LaTeX purists: This template doesn't use much of the structures LaTeX provides,
% such as sections and title. In the future there might be a LaTeX class for those reports
% helping with a common look and feel.
%
% All packages used should be fairly common and included in a standard LaTeX installation.
%
% A NOTE TO TEMPLATE DEVELOPERS:
%
% If using this template as starting point for a series of similar templates, it might be
% wise to use the "include" hook of the tpl template engine and to separate large parts of
% the header in an additional document that just gets added to all templates.
%
%%%%%%%%%%%%%%%%%%%%%%%%%%%%%%%%%%%%%%%%%%%%%%%%%%%%%%%%%%%%%%%%%%%%%%%%%%%%%%%%%%%%%%%%%%

% Copyright (c) 2015, Till Biskup, Deborah Meyer
% 2015-09-14

%%%%%%%%%%%%%%%%%%%%%%%%%%%%%%%%%%%%%%%%%%%%%%%%%%%%%%%%%%%%%%%%%%%%%%%%%%%%%%%%%%%%%%%%%%
% HEADER
%%%%%%%%%%%%%%%%%%%%%%%%%%%%%%%%%%%%%%%%%%%%%%%%%%%%%%%%%%%%%%%%%%%%%%%%%%%%%%%%%%%%%%%%%%
\documentclass{article}
\usepackage[paper=a4paper,left=25mm,right=25mm,top=25mm,bottom=20mm]{geometry}
\usepackage[utf8]{inputenc}
\usepackage[T1]{fontenc}

% For German reports, load appropriate language package
\usepackage[ngerman]{babel}

% Nice looking tables
\usepackage{booktabs}
% As "caption" shall be typeset above a table, no separation between caption and table.
\setlength{\abovetopsep}{1em}

% Math packages
\usepackage{amsmath}

% Graphics handling
\usepackage{graphicx}
  \DeclareGraphicsExtensions{.pdf,.jpg,.png}

% Headers and footers
\usepackage{fancyhdr}
\fancyhead[L]{Triplett-Simulation von PNDIT2 RM215.2 drop-cast}
\pagestyle{fancy}

% Change indentation and separation between paragraphs
\setlength{\parindent}{0ex}
\setlength{\parskip}{1ex}

% Define some basic commands
\newcommand{\matlab}{\textsf{Matlab}}

% Caption setup
\makeatletter

\setlength\abovecaptionskip{5\p@}
\setlength\belowcaptionskip{0\p@}
\@ifundefined{captionfontsize}{%
  \def\captionfontsize{\small} }{}
\long\def\@makecaption#1#2{%
  \vskip\abovecaptionskip\captionfontsize
  \sbox\@tempboxa{\textbf{#1:} #2}%
  \ifdim \wd\@tempboxa >\hsize
    \textbf{#1:} #2\par
  \else
    \global \@minipagefalse
    \hb@xt@\hsize{\hfil\box\@tempboxa\hfil}%
  \fi
  \vskip\belowcaptionskip}
  
\makeatother


%%%%%%%%%%%%%%%%%%%%%%%%%%%%%%%%%%%%%%%%%%%%%%%%%%%%%%%%%%%%%%%%%%%%%%%%%%%%%%%%%%%%%%%%%%
% BODY
%%%%%%%%%%%%%%%%%%%%%%%%%%%%%%%%%%%%%%%%%%%%%%%%%%%%%%%%%%%%%%%%%%%%%%%%%%%%%%%%%%%%%%%%%%
\begin{document}
\thispagestyle{empty}

\vspace*{-1.5cm}

\noindent\rule[1.5ex]{\textwidth}{1pt}

\begin{sffamily}\bfseries\large
Ergebnisse der Triplett-Simulation von PNDIT2 RM215.2 drop-cast
\end{sffamily}

\noindent\rule{\textwidth}{1pt}

\begin{flushright}\slshape
--- deborah, 2015-08-20 07:15:55
\end{flushright}

\vspace*{1.5em}

An das TREPR-Spektrum von PNDIT2 RM215.2 drop-cast wurde mittels des Programms \texttt{Tsim} und unter Verwendung der Simulationsroutine \texttt{pepper} eine Triplett-Simulation angepasst.


\textbf{Zielstellung:} sdf

\textbf{Nutzerkommentar:} hb


Für einen ersten Überblick über die Ergebnisse der Triplett-Simulation und der Anpassung an die experimentellen Daten vgl. Abb.~\ref{fig:ergebnisse}, für die zugehörigen Simulationsparameter Tab.~\ref{tab:simparameter}, S.~\pageref{tab:simparameter}, und für die Fitparameter Tab.~\ref{tab:fitparameter}, S.~\pageref{tab:fitparameter}.


\begin{figure}[h]
\centering

\includegraphics[width=\textwidth]{messung15_Sa71_85Grad_fit-fig}

\caption{\textbf{Gemessene Daten zusammen mit einer angepassten Simulation.} Angepasst wurde die Simulation an einen Schnitt bei 5.09e-07~s. Experimentelle Parameter: Mikrowellenfrequenz  9.70560~GHz, Mikrowellenleistung  2.01~mW (20~dB), 1500 spp, Lichtanregung bei 622~nm mit   1~mJ Pulsleistung. Für die Simulationsparameter vgl. Tab.~\ref{tab:simparameter}, S.~\pageref{tab:simparameter}.}
\label{fig:ergebnisse}
\end{figure}

\begin{center}
\setlength{\fboxsep}{1.5ex}\setlength{\fboxrule}{.75pt}
\fbox{%
\begin{minipage}{.9\textwidth}\centering
\textbf{--- Disclaimer der Simulationsroutine ---}
\vspace*{1ex}

\slshape

If you're using this routine for simulating your data,\par
please cite the publications you will find at:\par
\par
http://www.easyspin.org/\par
\end{minipage}
}
\end{center}

\clearpage


Für die Kurvenanpassung wurde die Funktion \texttt{lsqcurvefit} von \matlab{} verwendet. Die verwendeten Optionen können Tab.~\ref{tab:fitopt} entnommen werden. Das Abbruchkriterium der Kurvenanpassung lautete:

\begin{quote}\small
\ttfamily
Solver stopped prematurely.

lsqcurvefit stopped because it exceeded the iteration limit,
options.MaxIter = 20 (the default value).
\end{quote}

Gemessene Daten und Simulation wurden für die Kurvenanpassung jeweils auf gleiche Fläche normiert.

\begin{table}[h]
\caption{\textbf{Übersicht über die Simulationsparameter.} Die Simulationsparameter sind der vollständige Satz an Parametern, die für die Simulation des in Abb.~\ref{fig:ergebnisse} dargestellten Spektrums verwendet wurden. $\Gamma$ steht für die Linienbreite, mit der die Simulation gefaltet wurde. Welche Parameter für die Kurvenanpassung wie variiert wurden, kann der Tab.~\ref{tab:fitparameter} entnommen werden.}
\label{tab:simparameter}
\centering
\begin{tabular}{cccccccc} 
\toprule
\multicolumn{8}{c}{\textbf{Standardparameter} }
\\
\midrule 
$g_x$      & $g_y$      & $g_z$      & $p_1$   & $p_2$   & $p_3$   & $D$ / MHz & $E$ / MHz \\
2.00200 & 2.00200 & 2.00200 & 0.293 & 0.355 & 0.353 &   879.7  &   160.0     \\ 
\bottomrule
\end{tabular}

\begin{tabular}{cc}
\toprule
\textbf{Parameter} & \textbf{Wert}
\\
\midrule

$\Gamma_\text{Gauß}$ / mT & $5.60$
\\

$\Gamma_\text{Lorentz}$ / mT & $3.20$
\\

$\Delta B$ / mT & $-0.01$
\\

\bottomrule
\end{tabular}
\end{table}

\begin{table}[h]
\caption{\textbf{Übersicht über die Fitparameter.} Diese Parameter stellen in der Regel eine Untermenge der für die Simulation des in Abb.~\ref{fig:ergebnisse} dargestellten Spektrums verwendeten Parameter dar. Für einen vollständigen Satz der Simulationsparameter vgl. Tab.~\ref{tab:simparameter}. Der Fehler für jeden Parameter ist die Standardabweichung, die aus der Jacobi-Matrix berechnet wird.}
\label{tab:fitparameter}
\centering
\begin{tabular}{cccccc}
\toprule
\textbf{Parameter} & \textbf{Startwert} & \textbf{untere Grenze} & \textbf{obere Grenze} & \textbf{Endwert} & \textbf{Fehler}
\\
\midrule
%$g_x$ & 2.002 & 2.000 & 2.005 & 2.003

\texttt{D} & 600.00 & 500.00 & 750.00 & 879.68 & 10.40
\\
\texttt{E} & 150.00 & 140.00 & 160.00 & 160.00 & 5.78
\\
\texttt{p1} & 0.35 & 0.00 & 1.00 & 0.31 & 702433.81
\\
\texttt{p2} & 0.36 & 0.00 & 1.00 & 0.38 & 310581.30
\\
\texttt{p3} & 0.29 & 0.00 & 1.00 & 0.37 & 280584.12
\\
\texttt{lwGauss} & 2.00 & 0.00 & 7.00 & 5.60 & 2.08
\\
\texttt{lwLorentz} & 2.00 & 0.00 & 7.00 & 3.20 & 1.21
\\
\texttt{DeltaB} & 0.20 & -0.10 & 0.30 & -0.01 & 0.17
\\
\bottomrule
\end{tabular}
\end{table}

\begin{table}[h]
\caption{\textbf{Übersicht über die Optionen der \matlab-Routine \texttt{lsqcurvefit}, die bei der Kurvenanpassung Verwendung fanden.} Vergleiche auch das Abbruchkriterium der Funktion \texttt{lsqcurvefit} weiter oben.}
\label{tab:fitopt}
\centering
\begin{tabular}{lr}
\toprule
\textbf{Parameter} & \textbf{Wert}
\\
\midrule
%\texttt{TolFun} & 1e-18

\texttt{MaxIter} & 2.00e+01
\\
\texttt{MaxFunEval} & 1.00e+03
\\
\texttt{TolFun} & 1.00e-10
\\
\bottomrule
\end{tabular}
\end{table}


\end{document}
