%%%%%%%%%%%%%%%%%%%%%%%%%%%%%%%%%%%%%%%%%%%%%%%%%%%%%%%%%%%%%%%%%%%%%%%%%%%%%%%%%%%%%%%%%%
%
% TsimFitReport-de.tex - LaTeX template for reporting fit results from the Tsim toolbox.
%
%%%%%%%%%%%%%%%%%%%%%%%%%%%%%%%%%%%%%%%%%%%%%%%%%%%%%%%%%%%%%%%%%%%%%%%%%%%%%%%%%%%%%%%%%%
%
% The "-de" indicates that this is a GERMAN template. Feel free to translate into any
% other language while containing the general structure and contents.
%
% This template is meant to be used with the report functionality of the Tsim toolbox
% in combination with the common toolbox and the tpl template engine.
%
% Make sure to set the delimiters for fields to "[[" and "]]" accordingly, as it turned 
% out that the default "{{" and "}}" delimiters are not a very good idea for use with
% LaTeX.
%
% Please note that this template makes use of knowledge about some internals of the tpl
% template engine. It may serve as starting point for own templates for use with tpl, but
% be aware that tpl may still change quite dramatically before it reaches a stable
% interface.
%
% For LaTeX purists: This template doesn't use much of the structures LaTeX provides,
% such as sections and title. In the future there might be a LaTeX class for those reports
% helping with a common look and feel.
%
% All packages used should be fairly common and included in a standard LaTeX installation.
%
% A NOTE TO TEMPLATE DEVELOPERS:
%
% If using this template as starting point for a series of similar templates, it might be
% wise to use the "include" hook of the tpl template engine and to separate large parts of
% the header in an additional document that just gets added to all templates.
%
%%%%%%%%%%%%%%%%%%%%%%%%%%%%%%%%%%%%%%%%%%%%%%%%%%%%%%%%%%%%%%%%%%%%%%%%%%%%%%%%%%%%%%%%%%

% Copyright (c) 2015, Till Biskup
% 2015-06-20

%%%%%%%%%%%%%%%%%%%%%%%%%%%%%%%%%%%%%%%%%%%%%%%%%%%%%%%%%%%%%%%%%%%%%%%%%%%%%%%%%%%%%%%%%%
% HEADER
%%%%%%%%%%%%%%%%%%%%%%%%%%%%%%%%%%%%%%%%%%%%%%%%%%%%%%%%%%%%%%%%%%%%%%%%%%%%%%%%%%%%%%%%%%
\documentclass{article}
\usepackage[paper=a4paper,left=25mm,right=25mm,top=25mm,bottom=20mm]{geometry}
\usepackage[utf8]{inputenc}
\usepackage[T1]{fontenc}

% For German reports, load appropriate language package
\usepackage[ngerman]{babel}

% Nice looking tables
\usepackage{booktabs}
% As "caption" shall be typeset above a table, no separation between caption and table.
\setlength{\abovetopsep}{1em}

% Math packages
\usepackage{amsmath}

% Graphics handling
\usepackage{graphicx}
  \DeclareGraphicsExtensions{.pdf,.jpg,.png}

% Headers and footers
\usepackage{fancyhdr}
\fancyhead[L]{Triplett-Simulation von [[@sample.name]]}
\pagestyle{fancy}

% Change indentation and separation between paragraphs
\setlength{\parindent}{0ex}
\setlength{\parskip}{1ex}

% Define some basic commands
\newcommand{\matlab}{\textsf{Matlab}}

% Caption setup
\makeatletter

\setlength\abovecaptionskip{5\p@}
\setlength\belowcaptionskip{0\p@}
\@ifundefined{captionfontsize}{%
  \def\captionfontsize{\small} }{}
\long\def\@makecaption#1#2{%
  \vskip\abovecaptionskip\captionfontsize
  \sbox\@tempboxa{\textbf{#1:} #2}%
  \ifdim \wd\@tempboxa >\hsize
    \textbf{#1:} #2\par
  \else
    \global \@minipagefalse
    \hb@xt@\hsize{\hfil\box\@tempboxa\hfil}%
  \fi
  \vskip\belowcaptionskip}
  
\makeatother


%%%%%%%%%%%%%%%%%%%%%%%%%%%%%%%%%%%%%%%%%%%%%%%%%%%%%%%%%%%%%%%%%%%%%%%%%%%%%%%%%%%%%%%%%%
% BODY
%%%%%%%%%%%%%%%%%%%%%%%%%%%%%%%%%%%%%%%%%%%%%%%%%%%%%%%%%%%%%%%%%%%%%%%%%%%%%%%%%%%%%%%%%%
\begin{document}
\thispagestyle{empty}

\vspace*{-1.5cm}

\noindent\rule[1.5ex]{\textwidth}{1pt}

\begin{sffamily}\bfseries\large
Ergebnisse der Triplett-Simulation von [[@sample.name]]
\end{sffamily}

\noindent\rule{\textwidth}{1pt}

\begin{flushright}\slshape
--- [[@history{1}.system.username]], [[@history{1}.date]]
\end{flushright}

\vspace*{1.5em}

An das TREPR-Spektrum von [[@sample.name]] wurde mittels des Programms \texttt{Tsim} und unter Verwendung der Simulationsroutine \texttt{[[@TSim.sim.routine]]} eine Triplett-Simulation angepasst.

[[if ~isempty(this.assignments.TSim.remarks.purpose)]]
\textbf{Zielstellung:} [[@TSim.remarks.purpose]]
[[end]]

[[if ~isempty(this.assignments.TSim.remarks.comment)]]
\textbf{Nutzerkommentar:} [[@TSim.remarks.comment]]
[[end]]

Für einen ersten Überblick über die Ergebnisse der Triplett-Simulation und der Anpassung an die experimentellen Daten vgl. Abb.~\ref{fig:ergebnisse}, für die zugehörigen Simulationsparameter Tab.~\ref{tab:simparameter}, S.~\pageref{tab:simparameter}, und für die Fitparameter Tab.~\ref{tab:fitparameter}, S.~\pageref{tab:fitparameter}.


\begin{figure}[h]
\centering
\includegraphics[width=\textwidth]{[[@TSim.resultsFigure]]}
\caption{\textbf{Gemessene Daten zusammen mit einer angepassten Simulation.} Angepasst wurde die Simulation an einen Schnitt bei xx~s (gemittelt über yy~s). Experimentelle Parameter: Mikrowellenfrequenz [[@parameters.bridge.MWfrequency.value|f=8.5f]]~[[@parameters.bridge.MWfrequency.unit]], Mikrowellenleistung [[@parameters.bridge.power.value|f=5.2f]]~[[@parameters.bridge.power.unit]] ([[@parameters.bridge.attenuation.value|f=2.0f]]~[[@parameters.bridge.attenuation.unit]]), [[@parameters.recorder.averages|f=4.0f]] spp, Lichtanregung bei [[@parameters.laser.wavelength.value|f=3.0f]]~[[@parameters.laser.wavelength.unit]] mit [[@parameters.laser.power.value|f=3.0f]]~[[@parameters.laser.power.unit]] Pulsleistung. Für die Simulationsparameter vgl. Tab.~\ref{tab:simparameter}, S.~\pageref{tab:simparameter}.}
\label{fig:ergebnisse}
\end{figure}

\begin{center}
\setlength{\fboxsep}{1.5ex}\setlength{\fboxrule}{.75pt}
\fbox{%
\begin{minipage}{.9\textwidth}\centering
\textbf{--- Disclaimer der Simulationsroutine ---}
\vspace*{1ex}

\slshape
[[foreach @TSim.acknowledgement.sim]]
[[@TSim.acknowledgement.sim]]\par
[[end]]
\end{minipage}
}
\end{center}

\clearpage


Für die Kurvenanpassung wurde die Funktion \texttt{lsqcurvefit} von \matlab{} verwendet. Die verwendeten Optionen können Tab.~\ref{tab:fitopt} entnommen werden. Das Abbruchkriterium der Kurvenanpassung lautete:

\begin{quote}\small
\ttfamily
[[@TSim.fit.report.exitmessage]]
\end{quote}

Gemessene Daten und Simulation wurden für die Kurvenanpassung jeweils auf gleiche Fläche normiert. Zusätzlich kamen folgende Gewichtungen für bestimmte Bereiche des Spektrums zum Einsatz:

\begin{center}
\begin{tabular}{cc}
\toprule
\textbf{Bereich} & \textbf{Gewicht}
\\
\midrule
340--350 mT      & 0
\\
\bottomrule
\end{tabular}
\end{center}


\begin{table}[h]
\caption{\textbf{Übersicht über die Simulationsparameter.} Die Simulationsparameter sind der vollständige Satz an Parametern, die für die Simulation des in Abb.~\ref{fig:ergebnisse} dargestellten Spektrums verwendet wurden. $\Gamma$ steht für die Linienbreite, mit der die Simulation gefaltet wurde. Welche Parameter für die Kurvenanpassung wie variiert wurden, kann der Tab.~\ref{tab:fitparameter} entnommen werden.}
\label{tab:simparameter}
\centering
\begin{tabular}{cccccccc} 
\toprule
\multicolumn{8}{c}{\textbf{Standardparameter} }
\\
\midrule 
$g_x$      & $g_y$      & $g_z$      & $p_1$   & $p_2$   & $p_3$   & $D$ / MHz & $E$ / MHz \\
[[@TSim.sim.simpar.gx|f=7.5f]] & [[@TSim.sim.simpar.gy|f=7.5f]] & [[@TSim.sim.simpar.gz|f=7.5f]] & [[@TSim.sim.simpar.p1|f=5.3f]] & [[@TSim.sim.simpar.p2|f=5.3f]] & [[@TSim.sim.simpar.p3|f=5.3f]] & [[@TSim.sim.simpar.D|f=7.1f]]  & [[@TSim.sim.simpar.E|f=7.1f]]     \\ 
\bottomrule
\end{tabular}

\begin{tabular}{cc}
\toprule
\textbf{Parameter} & \textbf{Wert}
\\
\midrule
[[if isfield(this.assignments.TSim.sim.simpar,'lwGauss')]]
$\Gamma_\text{Gauß}$ / mT & $[[@TSim.sim.simpar.lwGauss]]$
\\
[[end]]
[[if isfield(this.assignments.TSim.sim.simpar,'lwLorentz')]]
$\Gamma_\text{Lorentz}$ / mT & $[[@TSim.sim.simpar.lwLorentz]]$
\\
[[end]]
[[if isfield(this.assignments.TSim.sim.simpar,'DeltaB')]]
$\Delta B$ / mT & $[[@TSim.sim.simpar.DeltaB]]$
\\
[[end]]
\bottomrule
\end{tabular}
\end{table}

\begin{table}[h]
\caption{\textbf{Übersicht über die Fitparameter.} Diese Parameter stellen in der Regel eine Untermenge der für die Simulation des in Abb.~\ref{fig:ergebnisse} dargestellten Spektrums verwendeten Parameter dar. Für einen vollständigen Satz der Simulationsparameter vgl. Tab.~\ref{tab:simparameter}. Der Fehler für jeden Parameter ist die Standardabweichung, die aus der Jacobi-Matrix berechnet wird.}
\label{tab:fitparameter}
\centering
\begin{tabular}{cccccc}
\toprule
\textbf{Parameter} & \textbf{Startwert} & \textbf{untere Grenze} & \textbf{obere Grenze} & \textbf{Endwert} & \textbf{Fehler}
\\
\midrule
%$g_x$ & 2.002 & 2.000 & 2.005 & 2.003
[[foreach element of @TSim.fit.fitpar]]
\texttt{[[@TSim.fit.fitpar([[@element]])]]} & [[@TSim.fit.initialvalue([[@element]])]] & [[@TSim.fit.lb([[@element]])]] & [[@TSim.fit.ub([[@element]])]] & [[@TSim.fit.finalvalue([[@element]])]] & [[@TSim.fit.report.stdDev([[@element]])]]
\\
[[end]]
\bottomrule
\end{tabular}
\end{table}

\begin{table}[h]
\caption{\textbf{Übersicht über die Optionen der \matlab-Routine \texttt{lsqcurvefit}, die bei der Kurvenanpassung Verwendung fanden.} Vergleiche auch das Abbruchkriterium der Funktion \texttt{lsqcurvefit} weiter oben.}
\label{tab:fitopt}
\centering
\begin{tabular}{lr}
\toprule
\textbf{Parameter} & \textbf{Wert}
\\
\midrule
%\texttt{TolFun} & 1e-18
[[foreach @TSim.fit.fitopt]]
\texttt{[[@TSim.fit.fitopt.fieldname]]} & [[@TSim.fit.fitopt.fieldvalue|f=4.2e]]
\\
[[end]]
\bottomrule
\end{tabular}
\end{table}


\end{document}